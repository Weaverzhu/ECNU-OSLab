\subsection{Objectives}

\begin{itemize}
    \item 使 xv6 支持简单的 thread api
    \item 实现两个系统调用 \texttt{int clone} 和 \texttt{int join}
    \item 实现两个库函数 \texttt{thread\_create} 和 \texttt{thread\_join}
\end{itemize}

\subsection{Steps}

我们实现的线程 \texttt{clone, join} 相当于进程中的 \texttt{fork, wait},并且


\subsection{Concept: something about x86}

回顾一下 x86 一些寄存器的用途

\subsubsection{通用寄存器 (GPR)}

\noindent
\begin{tabularx}{\textwidth}{|X|X|X|X|X|X|X|X|X|}
    \hline
    Register & Accumu-lator  &Base &Stack Pointer & Stack Base Pointer & Source & Desti-nation \\ \hline
    64-bit & RAX & RBX & RSP & RBP & RSI & RDI \\ \hline
    32-bit & EAX & EBX & ESP & EBP & ESI & EDI \\ \hline
    16-bit & AX & BX & SP & BP & SI & DI \\ \hline
\end{tabularx}

\subsubsection{指令寄存器 (IP)}

EIP 会指向下一个需要被执行的指令,如果没有分支的话。EIP 只有在 \texttt{call} 指令之后才会在堆栈上被读取

\subsection{Concept: CPU context}

一篇讲的不错的文章: 
https://samwho.dev/blog/2013/06/01/context-switching-on-x86/

回顾一下系统调用时上下文信息会存放在 trap frame 当中,在系统调用之后返回继续执行的时候会用这些信息重新组成上下文。所以我们如果要指定新的线程会进程(不管是在 \texttt{fork} 中还是 \texttt{clone} 中都需要设置 \texttt{proc->tf})

CPU 进行上下文切换的时候只会保留需要保留的状态,实现的方法是保存堆栈指针(stack pointer), 并且使用一个新的。当一个函数被调用的时候,EIP 也就是指令寄存器会指向下一条指令被存放的位置。如果我们需要开始执行 \texttt{start\_routine} 的话,就要把它赋值为函数指针所指向的地址

\subsection{Concept: calling function}

stack 从地址高位开始向地位扩展,其中 ebp 在高位界限,esp 在低位界限。存入栈的机制是,从高到低移动 esp 并在新的 esp 位置开始放入数据

c 语言默认使用 CDECL 调用方法,有如下特性

\begin{itemize}
    \item 如上所述参数从高位到地位放置
    \item 调用参数的函数会清理栈空间
\end{itemize}

\texttt{clone} 函数要求创建线程并直接开始运行 \texttt{start\_routine},观察 x86 如何调用进程:

\begin{asmcode}
    push eip + 2 ; return address is current address + size of two instructions
    jmp _MyFunction2
\end{asmcode}


\subsection{Concept: Thread \&  Process}

进程使操作系统虚拟化 CPU 的手段。在多程序同时运行的机器上,通过进程的实现使得每个程序都可以认为自己独占 CPU。

进程的信息被存放在 TCB (Thread Control Block) 中,而线程的信息被存放在 PCB 中 (Process Control Block)。但是,进程和线程有很类似的地方,他们都需要在调度器中被存储下关于上下文的东西,比如堆栈指针,一些寄存器的值,打开的文件(设备),上下文和运行状态等。

如果我们特意地去实现 TCB 和 PCB 并把它们区分开来,那么我们需要在 schedule 中耗费大量的精力。事实上进程和线程主要地区别在于派生出来的线程需要有额外自己的 stack frame,和父进程拥有相同的内存管理(page directory),有和父进程不一样的 pc。

所以这里我们用到的 trick 就是让 scheduler 认为进程和线程大抵是一样的东西(共用数据结构 \texttt{struct proc}),修改结果加上线程自己的栈空间 \texttt{void *thread\_stack} 然后指定 \texttt{eip, esp} 来规定 \texttt{start\_routine} 和自己的栈空间

\subsection{Code: clone \& join}



\subsection{Code: thread\_create \& thread\_join}

\subsubsection{why we should use lock}

\subsection{Test our thread api}