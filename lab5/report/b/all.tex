\subsection{Objectives}

\begin{itemize}
    \item 使 xv6 支持简单的 thread api
    \item 仿照 pthread 实现基本线程操作
    \item 实现两个系统调用 \texttt{int clone} 和 \texttt{int join},\texttt{clone} 会指定一个函数,及其参数和新线程应有的 \texttt{stack} 空间,然后 \texttt{clone} 为其创建线程,将控制流指向函数,然后放入调度器。\texttt{join} 会等待一个子线程结束运行并且返回得到该线程的栈,用以回收资源
    \item 实现两个库函数 \texttt{thread\_create} 和 \texttt{thread\_join},供用户使用
\end{itemize}

\subsection{Steps}

\begin{itemize}
    \item 我们实现的线程 \texttt{clone, join} 相当于进程中的 \texttt{fork, wait}。
    \item 线程有自己栈,但是共享内存空间和打开的设备和文件
    \item 创建线程和等待线程结束回收资源与 \texttt{fork, wait} 的区别主要在于需要对线程的栈进行操作
    \item 为线程栈分配空间的时候应该使用 \texttt{lock} 从而避免同时进入线程不安全的 \texttt{malloc}
\end{itemize}





\subsection{Concept: something about x86}

回顾一下 x86 一些寄存器的用途

\subsubsection{通用寄存器 (GPR)}

\noindent
\begin{tabularx}{\textwidth}{|X|X|X|X|X|X|X|X|X|}
    \hline
    Register & Accumu-lator  &Base &Stack Pointer & Stack Base Pointer & Source & Desti-nation \\ \hline
    64-bit & RAX & RBX & RSP & RBP & RSI & RDI \\ \hline
    32-bit & EAX & EBX & ESP & EBP & ESI & EDI \\ \hline
    16-bit & AX & BX & SP & BP & SI & DI \\ \hline
\end{tabularx}

\subsubsection{指令寄存器 (IP)}

EIP 会指向下一个需要被执行的指令,如果没有分支的话。EIP 只有在 \texttt{call} 指令之后才会在堆栈上被读取

\subsection{Concept: CPU context}

回顾一下系统调用时上下文信息会存放在 trap frame 当中,在系统调用之后返回继续执行的时候会用这些信息重新组成上下文。所以我们如果要指定新的线程会进程(不管是在 \texttt{fork} 中还是 \texttt{clone} 中都需要设置 \texttt{proc->tf})

CPU 进行上下文切换的时候只会保留需要保留的状态,实现的方法是保存堆栈指针(stack pointer), 并且使用一个新的。当一个函数被调用的时候,EIP 也就是指令寄存器会指向下一条指令被存放的位置。如果我们需要开始执行 \texttt{start\_routine} 的话,就要把它赋值为函数指针所指向的地址

\subsection{Concept: calling function}

x86 中 栈 (stack) 从地址高位开始向地位扩展,其中 ebp 在高位界限,esp 在低位界限。存入栈的机制是,从高到低移动 esp 并在新的 esp 位置开始放入数据

x86 执行一个函数的步骤如下:

在调用函数的地方:

\begin{enumerate}
    \item 反序将所有函数参数压入栈中
    \item 执行 \texttt{call} 指令,会把当前下一条指令压入栈中,然后修改 \texttt{eip} 的值,指向函数的第一条指令
\end{enumerate}

于是整个栈空间会变得像这样:

\begin{textcode}
    Parameter #N
    ...
    Parameter 2
    Parameter 1
    Return Address <--- (%esp)
\end{textcode}


在函数里面也有一些事情要做:

\begin{enumerate}
    \item 保存现在的 \texttt{ebp},并将其赋值给 \texttt{esp},这个过程相当于初始化一个栈空间给当前函数
\end{enumerate}

现在,ebp 往上是函数的参数,ebp往下就是函数自己的栈空间了

这个时候栈空间的样子:

\begin{textcode}
    Parameter #N <--- N*4+4(%ebp)
    ...
    Parameter 2 <--- 12(%ebp)
    Parameter 1 <--- 8(%ebp)
    Return Address <--- 4(%ebp)
    Old %ebp <--- (%esp) and (%ebp)
\end{textcode}

\begin{asmcode}
    subl $8, %esp
\end{asmcode}

之后函数将会给里面的本地变量预留出内存空间,方法就是扩大栈空间,将 \texttt{esp} 向低位移动。如果有两个参数的话那么栈空间是长成这个样子的:


\begin{textcode}
    Parameter #N <--- N*4+4(%ebp)
    ...
    Parameter 2 <--- 12(%ebp)
    Parameter 1 <--- 8(%ebp)
    Return Address <--- 4(%ebp)
    Old %ebp <--- (%ebp)
    Local Variable 1 <--- -4(%ebp)
    Local Variable 2 <--- -8(%ebp) and (%esp)
\end{textcode}

当函数返回的时候:

\begin{enumerate}
    \item 返回值存放在 \texttt{eax}
    \item 将栈回滚到调用这个函数的时候,也就是将 \texttt{ebp, esp} 重置为调用函数之前的值
    \item 使用 \texttt{ret} 指令将控制流送回,也就是将栈中存放的之前那个下一条指令的值放回 \texttt{eip}
\end{enumerate}

在 \texttt{clone} 中我们需要直接开始执行 \texttt{start\_routine} 函数,因此要由我们在 \texttt{clone} 系统调用中做好调用函数的工作


\subsection{Concept: Thread \&  Process}

进程使操作系统虚拟化 CPU 的手段。在多程序同时运行的机器上,通过进程的实现使得每个程序都可以认为自己独占 CPU。

进程的信息被存放在 TCB (Thread Control Block) 中,而线程的信息被存放在 PCB 中 (Process Control Block)。但是,进程和线程有很类似的地方,他们都需要在调度器中被存储下关于上下文的东西,比如堆栈指针,一些寄存器的值,打开的文件(设备),上下文和运行状态等。

如果我们特意地去实现 TCB 和 PCB 并把它们区分开来,那么我们需要在 schedule 中耗费大量的精力。事实上进程和线程主要地区别在于派生出来的线程需要有额外自己的 stack frame,和父进程拥有相同的内存管理(page directory),有和父进程不一样的 pc。

所以这里我们用到的 trick 就是让 scheduler 认为进程和线程大抵是一样的东西(共用数据结构 \texttt{struct proc}),修改结果加上线程自己的栈空间 \texttt{void *thread\_stack} 然后指定 \texttt{eip, esp} 来规定 \texttt{start\_routine} 和自己的栈空间

\subsection{Code: clone \& join}



\subsection{Code: thread\_create \& thread\_join}

\subsubsection{why we should use lock}

\subsection{Test our thread api}